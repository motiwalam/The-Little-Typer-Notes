\documentclass{article}

\usepackage{amsmath, amsthm, amssymb, amsfonts, mathtools}
\usepackage{thmtools}
\usepackage{graphicx}
\usepackage{setspace}
\usepackage[margin=0.7in]{geometry}
\usepackage{float}
\usepackage{hyperref}
\usepackage[utf8]{inputenc}
\usepackage[english]{babel}
\usepackage{framed}
\usepackage[dvipsnames]{xcolor}
\usepackage{tcolorbox}
\usepackage{minted}
\usepackage{etoc}
\usepackage{etoolbox}
\AtBeginEnvironment{appendix}{\etocsettocdepth.toc{section}\etocignoretoctocdepth}

\hypersetup{
    colorlinks,
    linkcolor={blue!80!black},
    citecolor={blue!80!black},
    urlcolor={blue!80!black}
}

\colorlet{LightGray}{White!90!Periwinkle}
\colorlet{LightOrange}{Orange!15}
\colorlet{LightGreen}{Green!15}

\newcommand{\HRule}[1]{\rule{\linewidth}{#1}}
\newcommand{\ttt}[1]{\texttt{#1}}
\newcommand{\bb}[1]{\mathbb{#1}}
\newcommand{\bN}{\bb{N}}

\newcommand{\lo}{\mathrm{lo}}

\declaretheoremstyle[name=Theorem,]{thmsty}
\declaretheorem[style=thmsty,numberwithin=section]{theorem}
\tcolorboxenvironment{theorem}{colback=LightGray}

\declaretheoremstyle[name=Proposition,]{prosty}
\declaretheorem[style=prosty,numberlike=theorem]{proposition}
\tcolorboxenvironment{proposition}{colback=LightOrange}

\declaretheoremstyle[name=Principle,]{prcpsty}
\declaretheorem[style=prcpsty,numberlike=theorem]{principle}
\tcolorboxenvironment{principle}{colback=LightGreen}

\setstretch{1.2}
\geometry{
    textheight=9in,
    textwidth=5.5in,
    top=1in,
    headheight=12pt,
    headsep=25pt,
    footskip=30pt
}

% ------------------------------------------------------------------------------

\begin{document}

% ------------------------------------------------------------------------------
% Cover Page and ToC
% ------------------------------------------------------------------------------

\title{ \normalsize \textsc{}
		\\ [2.0cm]
		\HRule{1.5pt} \\
		notes on \LARGE \textbf{\uppercase{The Little Typer}
		\HRule{2.0pt} \\ [0.6cm] \LARGE{CSC392} \vspace*{10\baselineskip}}
		}
\date{}
\author{\textbf{Mustafa Motiwala} \\ 
		1009298868}

\maketitle
\newpage

\tableofcontents
\newpage

% ------------------------------------------------------------------------------

\section{The More Things Change, the More They Stay the Same}
\subsection{Summary}
This chapter introduces terminology, focusing on \textit{judgements} (propositional statements of a specific shape), \textit{expressions} (every well-formed statement about which it makes sense to make judgements), and \textit{types}.

\subsection{The \ttt{Atom} type}
\begin{itemize}
    \item infinitely many constructors! (each atom constructs itself)
    \item an atom is a tick-mark followed by a sequence of letters and hyphens
    \begin{itemize}
        \item \ttt{'a, 'b, 'a-b-c} are atoms
        \item \ttt{'0, 'a0, ', ''} are \textbf{not} atoms
    \end{itemize}
\end{itemize}

\subsection{Judgements}
\begin{itemize}
    \item Has \textbf{four} forms
        \begin{enumerate}
            \item \underline{\phantom{blank}} is a \underline{\phantom{blank}}.
                \begin{itemize}
                    \item \ttt{'a} is a \ttt{Atom}.
                    \item \ttt{'0} is a \ttt{Atom} (not true but still a judgement).
                \end{itemize}
            \item \underline{\phantom{blank}} is the same \underline{\phantom{blank}} as \underline{\phantom{blank}}.
                \begin{itemize}
                    \item \ttt{'a} is the same \ttt{Atom} as \ttt{'a}.
                    \item \ttt{'b} is the same \ttt{Atom} as \ttt{(cdr (cons 'a 'b))}.
                \end{itemize}
            \item \underline{\phantom{blank}} is a type.
                \begin{itemize}
                    \item \ttt{Atom} is a type.
                    \item \ttt{(Pair Atom Atom)} is a type.
                \end{itemize}
            \item \underline{\phantom{blank}} and \underline{\phantom{blank}} are the same type.
                \begin{itemize}
                    \item \ttt{Atom} and \ttt{Atom} are the same type.
                \end{itemize}
        \end{enumerate}
    \item Can be believable (true) or not believable (false).
    \item Some judgements require other judgements (i.e, presuppositions) to even make sense.
        \begin{itemize}
            \item the judgment ``\ttt{'a} is the same \ttt{'b} as \ttt{'c}" requires first the judgement that ``\ttt{'b} is a type"
        \end{itemize}
\end{itemize}

\subsection{Normal Forms}
\begin{itemize}
    \item The \textit{normal form} of an expression is ``the most direct" (\textcolor{red}{how can we formalize this?}) way of writing that expression.
    \item Allows us to define equality on expressions. Two expressions are the same if and only if their normal forms are the same.
        \begin{itemize}
            \item \textbf{Note:} Sameness is a judgement that is different for types and non-types.
            \item If expressions \(A\) and \(B\) are types, then they are the same type (judgement 4) iff their normal forms are identical.
            \item If expressions \(A\) and \(B\) are not types but are themselves described by a type \(T\) (judgement 1), then \(A\) and \(B\) are the same iff their normal forms (with respect to \(T\)) are identical.
        \end{itemize}
\end{itemize}

\subsection{Constructors \& Type Constructors}
\begin{itemize}
    \item A type is defined by describing its constructors; \textbf{constructor expressions are the direct ways of building expressions with that type}
        \begin{itemize}
            \item \ttt{zero} and \ttt{add1} are the constructors of the \ttt{Nat} type.
            \item \ttt{cons} is the constructor of \ttt{Pair} types.
            \item Each atom constructs itself.
        \end{itemize}
    \item type constructors construct types.
        \begin{itemize}
            \item \ttt{Pair} is a type constructor; \ttt{Pair} on its own is not a type - it needs two arguments to define a new type.
        \end{itemize}
\end{itemize}

\subsection{Values}
\begin{itemize}
    \item A value is an expression with a constructor at the top.
    \item Since constructor arguments do not have to be normal, not all values are normal.
    \item Finding a value of an expression is known as evaluation.
    \item Since values don't have to be normal, one expression can have multiple values.
    \item (\textbf{not in book}) normal forms (of non-type expressions) are values.
\end{itemize}

\subsection{Claims and Definitions}
\begin{itemize}
    \item We associate a name with an expression via \ttt{define}
    \begin{itemize}
        \item \ttt{(define two (add1 (add1 zero)))}
    \end{itemize}
    \item Names must first be given types via \ttt{claim} \textit{before defining}.
    \begin{itemize}
        \item \ttt{(claim two Nat)}
    \end{itemize}
    \item These are forever. Names can be \ttt{claim}ed and \ttt{define}d at most once.
\end{itemize}

\section{Doin' What Comes Naturally}
\subsection{Summary}
This chapter focuses heavily on \textit{eliminators} which are opposed to \textit{constructors} that are discussed in the previous chapter.
\\ \\
We discuss the \(\lambda\) constructor and its eliminator, the application operator (juxtaposition in Pie). The \ttt{which-Nat} eliminator for the \ttt{Nat} type is also introduced.
\\ \\
Recursion is not an option in Pie.
\\ \\
We also discuss the \(\mathcal{U}\) type, which is the type of types. Every \(\mathcal{U}\) is a type, but not every type is a \(\mathcal{U}\); this is because \(\mathcal{U}\) is a type but is not a \(\mathcal{U}\) (no type is itself).
\\ \\
Other trivial things like the semantics of function application (substitution), ``neutral" expressions (expressions with free variables), and more on pairs, \ttt{cons}, \ttt{car}, and \ttt{cdr}.
\subsection{Constructors and Eliminators}
\begin{itemize}
    \item Constructors \textit{build} values whereas Eliminators \textit{dismantle} values.
        \begin{itemize}
            \item \ttt{(cons 1 2)} builds a \ttt{(Pair Nat Nat)}; to then access the numbers \ttt{1} and \ttt{2} again, we use \ttt{car} and \ttt{cdr}.
            \item \ttt{(\(\lambda\) (s) s)} builds a \ttt{(\(\rightarrow\) Atom Atom)}; application is the eliminator for functions
        \end{itemize}
    \item The application eliminator for functions works by pure substitution. Hence, if \ttt{f} is a \ttt{(\(\rightarrow\) X Y)} then \ttt{(\(\lambda\) (y) (f y))} is the same \ttt{(\(\rightarrow\) X Y)} as \ttt{f} \textit{as long as \ttt{y} does not appear in \ttt{f}}. \textcolor{red}{Is this last condition really necessary in Pie, or have the authors introduced it just to make teaching the semantics of function application simpler?}
    \item The eliminator for \ttt{Nat} is \ttt{which-Nat} which takes a \ttt{Nat} \(k\), an expression which is returned if \(k = 0\) and a function \ttt{f} which, if \(k = \ttt{(add1 n)}\) is equal to \ttt{(f n)}. So, \[
        \ttt{(\(\lambda\) (k) (which-Nat k zero add1))}
    \]
    is a silly way of implementing the identity on \ttt{Nat}. \textcolor{red}{How do we write the type of \ttt{which-Nat}? What is the syntax for type variables?}
    \item \textbf{Note:} in languages like Haskell and Erlang that support pattern matching, it seems like we get eliminators built in to the syntax of the language. \textcolor{red}{How do Pie eliminators differ from pattern matching?}
\end{itemize}
\subsection{Type Values and the Universal Type \(\mathcal{U}\)}
\begin{itemize}
    \item \ttt{\(\mathcal{U}\)} is the universal type; it is the type of types.
    \item To define a type alias, we use \ttt{(claim alias \(\mathcal{U}\))} and \ttt{(define alias type-expr)}
        \begin{itemize}
            \item \ttt{(claim Predicate \(\mathcal{U}\))} and \ttt{(define Predicate (\(\rightarrow\) Nat Bool)}
        \end{itemize}
    \item Type constructors are parameterized types, such as \ttt{Pair} or \ttt{\(\rightarrow\)}.
    \item Type values are expressions with a type constructor at the top.
    \item Every \(\mathcal{U}\) is a type; but not all types are \(\mathcal{U}\) (in particular, \(\mathcal{U}\) is a type but not a \(\mathcal{U}\) because no type is itself)
\end{itemize}
\subsection{Equivalence of expressions}
\begin{itemize}
    \item Neutral expressions are those with free variables; identically written neutral expressions are always the same, regardless of type.
    \item The semantics of \ttt{claim} and \ttt{define} are as you'd expect; if \ttt{name} is \ttt{claim}ed to be an \ttt{X} and \ttt{define}d to be \ttt{expr}, then \ttt{name} is the same \ttt{X} as \ttt{expr}.
    \item \(\eta\)-equivalence; \ttt{f} is the same function as \ttt{(\(\lambda\) (x) (f x))}.
    \item If \ttt{p} is a \ttt{(Pair X Y)} then \ttt{p} is the same \ttt{(Pair X Y)} as \ttt{(cons (car p) (cdr p))}.
\end{itemize}
\section{Eliminate All Natural Numbers!}
\subsection{Summary}
This chapter is all about two new eliminators for \ttt{Nat}, \ttt{iter-Nat} and \ttt{rec-Nat}.
\\ \\
These solve the main problem with \ttt{which-Nat}, which is that it only eliminates one constructor at a time. That is, if \ttt{n = (add1 k)}, then \ttt{which-Nat} will eliminate the singular \ttt{add1} but will not eliminate \ttt{k}. In contrast, \ttt{iter-Nat} and \ttt{rec-Nat} will eliminate all the way down to \ttt{zero}.
\\ \\
\ttt{iter-Nat} takes a natural number \(n\), an initial value \(s\) of type \(X\), and a unary function \(f : X \to X\), and returns \(f^n(s)\) where \(f^n\) is the \(n\)-fold composition.
\\ \\
\ttt{rec-Nat} takes a natural number \(n\), an initial value \(s\), and a binary function \(f : \mathbb{N} \times X \to X\) and returns \[
    f(n - 1, f(n-2, f(\dots, f(0, s)\dots)))
\]
The chapter ends with a note on currying: all functions in Pie are functions of one variable and those functions which look multivariable are in fact just single variable functions whose return type is a new function.
\subsection{Totality} 
A function is \textit{total} if it assigns a value to every possible argument. Totality is built into the ``typical" notion of functions from mathematics, as a function \(f : A \to B\) is not a function if \(f(a)\) is not defined for some element \(a \in A\). In Pie, \textit{all functions are total.}
\subsection{The \ttt{iter-Nat} eliminator}
For \ttt{X} a type, \ttt{target} a \ttt{Nat}, \ttt{base} a \ttt{X}, and \ttt{step} a function from \ttt{X} to \ttt{X}, \[ E \coloneqq \ttt{(iter-Nat target base step)} \] is an \ttt{X}. If \ttt{target} is zero, then \(E\) is the same \ttt{X} as \ttt{base}. Otherwise, if \(E = \) \ttt{(add1 n)}, then \(E\) is the same \ttt{X} as \[
    \ttt{(step (iter-Nat n base step))}
\]
Intuitively, \ttt{iter-Nat} iterates the function \ttt{step} \ttt{target} times on \ttt{base}.
\subsection{The \ttt{rec-Nat} eliminator} \label{rec-nat-elim}
For \ttt{X} a type, \ttt{target} a \ttt{Nat}, \ttt{base} a \ttt{X}, and \ttt{step} a binary function of \ttt{Nat} and \ttt{X} to \ttt{X}, \[ E \coloneqq \ttt{(rec-Nat target base step)} \] is an \ttt{X}. If \ttt{target} is zero, then \(E\) is the same \ttt{X} as \ttt{base}. Otherwise, if \(E = \) \ttt{(add1 n)}, then \(E\) is the same \ttt{X} as \[
    \ttt{(step n (rec-Nat n base step))}
\]
\ttt{rec-Nat} achieves \href{https://en.wikipedia.org/wiki/Primitive_recursive_function}{primitive recursion} on the naturals. I don't really know what this means yet, but it looks cool!
\subsection{Superfluence of \ttt{which-Nat} and \ttt{iter-Nat} with respect to \ttt{rec-Nat}}
\ttt{rec-Nat} can be used to implement \ttt{which-Nat} and \ttt{iter-Nat} by using a function which ignores the second and first arguments respectively. That is, \[
    \ttt{(which-Nat n s f)} = \ttt{(rec-Nat n s (\(\lambda\) (n-1 \_) (f n-1)))}
\]
and \[
    \ttt{(iter-Nat n s f)} = \ttt{(rec-Nat n s (\(\lambda\) (\_ f-n-1) (f f-n-1)))}
\]
\subsection{Superfluence of \ttt{rec-Nat} with respect to \ttt{iter-Nat}} \label{recNat-in-iterNat}
\ttt{iter-Nat} can be used to implement \ttt{rec-Nat}. Let \(X\) be a set, \(f : \mathbb{N} \times X \to X\) a function. Define \(g : \mathbb{N} \times X \to \mathbb{N} \times X\) by \[
    g(n, x) = (n + 1, f(n, x))
\]
Then, for all \(n \in \bN\), \(s \in X\), \(g^n(0, s) = (n, \ttt{rec-Nat}(n, s, f))\).
\\ \\
We can see this by induction on \(n\). Let \(s \in X\). In the base case, \[
    g^0(0, s) = (0, s) = (0, \ttt{rec-Nat}(0, s, f))
\]
Now, let \(k \in \bN\) be arbitrary and suppose \(g^k(0, s) = (k, \ttt{rec-Nat}(k, s, f))\). Then,
\begin{align*}
    g^{k + 1}(0, s) 
        &= g(g^k(0, s)) \\
        &= g(k, \ttt{rec-Nat}(k, s, f)) \\
        &= (k + 1, f(k, \ttt{rec-Nat}(k, s, f))) \\
        &= (k + 1, \ttt{rec-Nat}(k + 1, s, f))
\end{align*}
Of course, \(g^n(0, s)\) is given by \ttt{(iter-Nat n (cons 0 s) g)}, which gives the result.
\\ \\
This implementation of \ttt{rec-Nat} in terms of \ttt{iter-Nat} can be found \hyperref[code:rec-Nat2]{here}.
\subsection{Some notes on primitive recursion}
As mentioned in \autoref{rec-nat-elim}, the \ttt{rec-Nat} eliminator achieves primitive recursion.
\begin{itemize}
    \item 
        Primitive recursion \textbf{does not} yield every possible computable function. A cool counterexample is the \href{https://en.wikipedia.org/wiki/Ackermann_function}{Ackermann function} which is well-defined, total, and computable, but not primitive recursive. \textcolor{red}{Does this mean that the Ackermann function can not be defined in Pie? Does Pie have more constructs beyond \ttt{rec-Nat} capable of surpassing the computing power of primitive recursion?}
    \item
        An intuitive way of understanding the computing power of primitive recursion is via ``the computer language" definition. That is, primitive recursion has the same computing power as a programming language with arithmetic, conditionals, comparisons, and \textbf{bounded} loops.
    \item
        Clearly, \ttt{rec-Nat} allows us to define functions that depend only on their previous value. That is, if \(f : \mathbb{N} \to \mathbb{N}\) is a function such that \(f(n + 1)\) is defined in terms of \(f(n)\) only, then it is easy to implement this using \ttt{rec-Nat} (\textcolor{red}{Is this a sufficient condition for primitive recursion? Do there exist functions like this that are not primitive recursive?}) What if \(f\) depends on two or more previous values, such as the Fibonacci function? We can use something called \href{https://en.wikipedia.org/wiki/Course-of-values_recursion}{course of values recursion} to define these.
    \item
        As a concrete example of the above, here is how we might encode the function \(F : \bN \to \bN\) defined by \[
            F(n) = \begin{cases}
                        1 & n < 2 \\
                        F(n - 1) + F(n - 2) & n \geq 2
                   \end{cases}
        \]
        into a function \(g : \bN \to \bN\) so that \(g(n + 1)\) depends only on \(g(n)\). 
        \\ \\
        We define \(g(n) = 2^{F(n)}3^{F(n + 1)}\). Then, \(g(0) = 6\) and \begin{align*}
            g(n + 1) &= 2^{F(n + 1)}3^{F(n + 2)} \\
                     &= 2^{\lo(3, g(n))}3^{F(n + 1) + F(n)} \\
                     &= 2^{\lo(3, g(n))}3^{F(n + 1)} 3^{F(n)} \\
                     &= 2^{\lo(3, g(n))}3^{\lo(3, g(n))} 3^{\lo(2, g(n))}
        \end{align*}
        Here, the \(\lo(a, b)\) function gives the number of times \(b\) is divisible by \(a\). \hyperref[code:lo]{This function}, as well as \hyperref[code:exponentiation]{exponentiation} and \hyperref[code:multiplication]{multiplication} can be defined using primitive recursion, and so \(g\) can be defined using primitive recursion, and so \(F(n) = \lo(2, g(n))\) can be \hyperref[code:Fibonacci]{defined via primitive recursion}, and hence the Fibonacci function is primitive recursive!  
\end{itemize}
\section{Easy as Pie}
\subsection{Summary}
This chapter is all about the \(\Pi\) type constructor, which allows us to abstract functions over types. With the \(\Pi\) type constructor, we can define functions whose return type depends on the types of its arguments. 
\subsection{The \(\Pi\) Constructor}
The \(\Pi\) constructor has the form \[
    \ttt{(\(\Pi\) ((x\(_1\) X\(_1\)) ... (x\(_n\) X\(_n\))) R)}
\]
where each \(\ttt{x}_i\) is an identifier, \(\ttt{X}_i\) is a type, and \ttt{R} is an expression describing a type, possible using the identifiers \(\ttt{x}_i\) (where \(\ttt{x}_i\) represents a value of type \(\ttt{X}_i\)).
\\ \\
A value of type \ttt{(\(\Pi\) ((x\(_1\) X\(_1\)) ... (x\(_n\) X\(_n\))) R)} is a function which takes \(n\) arguments, where the \(i\)'th argument is of type \(\ttt{X}_i\) and produces a result of type \ttt{R}.
\subsection{Superfluence of \(\to\) with respect to \(\Pi\)}
The \(\to\) type constructor is syntax sugar for the \(\Pi\) type constructor. The type \[
    \ttt{(\(\to\) X\(_1\) ... X\(_n\) R)}
\] is equivalent to the type \[
    \ttt{(\(\Pi\) ((x\(_1\) X\(_1\)) ... (x\(_n\) X\(_n\))) R)}
\]
The power of the \(\Pi\) operator comes from the fact that the \ttt{R} type is allowed to depend on the \ttt{x}\(_i\)'s, so if \ttt{X}\(_i\) is \(\mathcal{U}\), then \ttt{R} can denote a variable type.
\section{Lists, Lists, and More Lists}
\subsection{Summary}
This chapter introduces the \ttt{List} type, which behaves about how you'd expect. There isn't really anything new or surprising here.
\subsection{The \ttt{List} type}
Given a type \ttt{E}, \ttt{(List E)} describes a new type.
\subsection{The \ttt{nil} and \ttt{::} constructors}
The \ttt{List} type has two constructors, \ttt{nil} and \ttt{::}; the first constructs the empty list whereas the second tacks on a new element ``to the front" of an existing list.
\subsection{The \ttt{rec-List} eliminator}
Given a list \((x_1, \dots, x_n)\) of type \ttt{(List A)}, some value \(s\) of type \(X\), and a function \(f : \ttt{A} \times \ttt{(List A)} \times X \to X\), the \ttt{rec-List} eliminator returns \[
    f(x_1, (x_2, \dots, x_n), f(x_2, (x_3, \dots, x_n), f(\dots, f(x_n, (), s) \dots)))
\]
In other words, let \ttt{target} be a \ttt{(List E)}, \ttt{base} be an \ttt{X}, and \ttt{step} be a \ttt{(\(\to\) E (List E) X X)}. If \ttt{target} is the same as \ttt{nil}, then \ttt{(rec-List target base step)} is the same \ttt{X} as \ttt{base}. If \ttt{target} is \ttt{(:: e es)}, then \ttt{(rec-List target base step)} is the same \ttt{X} as \[
    \ttt{(step e es (rec-List es base step))}
\]
\ttt{rec-List} is primitive recursion on lists.
\subsection{Some questions about \ttt{rec-List}}
\subsubsection{Implementing lists with \ttt{Nat}?}
Given functions \(\ttt{encode} : X \to \bN\) and \(\ttt{decode} : \bN \to X\), we can represent a \ttt{(List X)} \((x_0, \dots, x_n)\) as the natural number \[
    \prod_{i = 0}^n p_i^{\ttt{encode}(x_i)}
\] where \(p_i\) is the \(i\)'th prime number.
\\ \\
So, \textcolor{red}{might we implement the \ttt{List} type just using \ttt{Nat}? What might go wrong?}
\\ \\
Probably, a true equivalent to the \ttt{List} type is not possible, as writing the \ttt{encode} and \ttt{decode} functions for \ttt{Atom} is not possible. However, encoding and decoding functions are easy to write for \ttt{Nat} (successor \& predecessor) and types built on \ttt{Nat} (products of powers of primes \& \hyperref[code:lo]{\(p\)-adic valuations}), so we can at least consider lists of these types.
\\ \\
We can represent the empty list with \(0\) so that the length of a list \(\ell\) is 0 if \(\ell = 0\) and otherwise \(\min \{ n \in \bN : p_{n + 1} \nmid \ell \}\) which can be implemented using the \hyperref[code:mu]{bounded minimization operator} and the \hyperref[code:nth-prime]{nth-prime} function.
\\ \\
Then, \ttt{rec-List} should also be implementable by using \ttt{rec-Nat} with the target being the length of the list and the step function using \hyperref[code:nth-prime]{nth-prime} and \hyperref[code:lo]{lo} to extract the correct value of \ttt{e} and dividing by all the previous primes to get the value of \ttt{es}.
\subsubsection{The second argument in the step of \ttt{rec-List}}
The second argument of the \ttt{step} parameter of \ttt{rec-List} takes both the first element in the list and the remaining elements, as well as the ``almost result". \textcolor{red}{What is the purpose of this argument? Are there functions that can not be implemented without it?}
\\ \\
For one thing, the presence of this argument can be made sense of by an appeal to symmetry with \ttt{rec-Nat}. \ttt{Nat} and \ttt{List} are both very similar, in that they are both constructed by starting with a ``base" constructor of zero arguments and then repeatedly applying a sort of ``successor" constructor. \ttt{which-Nat} removes one instance of this successor constructor and gives you access to what's underneath, whereas \ttt{iter-Nat} eliminates ``all the way down" to the base constructor and gives you access to the partial results. \ttt{rec-Nat} combines the power of these two by giving the step function both what is immediately underneath the successor constructor and the partial result.
\\ \\
For \ttt{List}, the successor constructor takes two arguments, and so \ttt{rec-List} gives the step function both of these arguments, which are the head and the tail of the list.
\\ \\
This isn't exactly illuminating though. The question remains, \textcolor{red}{is this argument actually useful?} Now, I think the second argument is not \textit{strictly} necessary as long as you have the first one, since you can build it up from scratch, by using a similar trick as in \autoref{recNat-in-iterNat}. Again, though, what kind of algorithms \textit{require} this argument (whether it is explicitly passed in or built up from scratch)?
\section{Precisely How Many?}
\subsection{Summary}
This chapter gives us our first taste of dependent types. The \ttt{Vec} type describes lists with length, so that it depends not only on a type, but on a \textit{value} as well.
\\ \\
We see two eliminators for the \ttt{Vec} type, \ttt{head} and \ttt{tail}, and discuss how to write down the types of these eliminators so that they are total.
\subsection{The \ttt{Vec} type}
Given a type \ttt{E}, and a \ttt{Nat k}, \ttt{(Vec E k)} is a type which describes a list of elements of type \ttt{E} with length \ttt{k}.
\subsection{The \ttt{vecnil} and \ttt{vec::} constructors}
The \ttt{vecnil} constructor constructs a \ttt{(Vec E 0)}.
\\ \\
The \ttt{vec::} constructor takes a value of type \ttt{E} and a \ttt{(Vec E k)} and constructs a \ttt{(Vec e (add1 k))}.
\subsection{The \ttt{head} and \ttt{tail} eliminators}
The \ttt{head} and \ttt{tail} eliminators do what they sound like they do; \ttt{head} returns the first element whereas \ttt{tail} drops the first element and returns the rest. These are natural functions to define on lists, but they can not be defined for any \ttt{List} type, because the expressions \ttt{(head nil)} and \ttt{(tail nil)} are meaningless. 
\\ \\
Clearly, \ttt{head} and \ttt{tail} are only meaningful when applied to non-empty lists; so we somehow need to describe a type that excludes empty lists! This is where \(\Pi\) comes in; \[
    \text{\ttt{head} has type \ttt{(\(\Pi\) ((a U) (k Nat)) (\(\to\) (Vec a (add1 k)) a))}}
\] and \[
    \text{\ttt{tail} has type \ttt{(\(\Pi\) ((a U) (k Nat)) (\(\to\) (Vec a (add1 k)) (Vec a k)))}}
\]
The type of \ttt{head} depends on a natural number \(k\); given a \(k\), \ttt{head} accepts a vector of length \(k + 1\), thereby guaranteeing the existence of at least one element in the vector and thus making \ttt{head} total. Similarly for \ttt{tail}.
\subsection{Some initial impressions}
Dependent types give us \textit{a lot} more power to describe programs. Suddenly, we can make Pie a lot more ``dynamic". For example, consider the following Python program:
\begin{minted}{python}
    apply = lambda f, args: f(*args)
\end{minted}
It is clear what \ttt{apply} is \textit{supposed} to do: take a function and some collection of arguments, and apply the function to those arguments. It is also clear that this function can easily raise an error, but that's par for the course in Python. However, in Pie, we can actually give a reasonable type for \ttt{apply}: \[
    \ttt{(\(\Pi\) ((a U) (k Nat)) (\(\to\) (NaryOp k a) (Vec a k) a))}
\] where \ttt{NaryOp} is as \hyperref[code:NaryOp]{defined here}. 
\\ \\
Of course, the above type doesn't fully describe the type of our Python program \ttt{apply}, but it's pretty good!
\\ \\
Another cool (but admittedly silly) use case is defining \ttt{length}, which we can simply define as:
\begin{minted}{scheme}
(claim length (Pi ((a U) (k Nat)) (-> (Vec a k) Nat)))
(define length (lambda (a k v) k))
\end{minted}
Similarly, we can define more expressive types for familiar list functions such as \hyperref[code:map]{\ttt{map}} and \hyperref[code:reverse]{\ttt{reverse}} which would express the fact that these do not affect the length of the input list, or \hyperref[code:repeat]{\ttt{repeat}}, whose type would reflect the fact that the length of the output list is the same as the input natural number.
\\ \\
Unfortunately, we can't implement a lot of these quite yet, since the tools we have for iteration/recursion as of right now will not work for \ttt{Vec}.
\subsection{How far can we go with dependent types?}
In this chapter, we saw how the types of \ttt{head} and \ttt{tail} involve the term \ttt{(Vec a (add1 k))}. That \ttt{(add1 k)} can be replaced with any arbitrary program that produces a natural number; so, \textcolor{red}{how can Pie tell if two of these types of expressions are equal if they are written in different ways?}
\\ \\
For example, suppose we have a function \[
    \ttt{(claim double (\(\Pi\) ((a U) (k Nat)) (\(\to\) (Vec a k) (Vec a (+ k k)))))}
\]
and a function \[
    \ttt{(claim un-interleave (\(\Pi\) ((a U) (k Nat)) (\(\to\) (Vec a (* 2 k)) (Pair (Vec a k) (Vec a k)))))}
\]
Would we be able to compose these functions? How can we tell Pie that \ttt{(+ k k)} is the same \ttt{Nat} as \ttt{(* 2 k)}?
\section{It All Depends on the Motive}
\subsection{Summary}
This chapter introduces the \ttt{ind-Nat} eliminator, which allows us to use \ttt{rec-Nat} in situations where the type of the ``almost results" change with each call to the step function. 
\\ \\
This opens up a new class of programs that we couldn't implement before, such as programs that construct or eliminate vectors. In general, \ttt{ind-Nat} enables us to write recursive programs that operate on values of dependent types.
\subsection{The \ttt{ind-Nat} eliminator}
For
\begin{align*}
    \ttt{target} &\quad \text{a \ttt{Nat}} \\
    \ttt{mot} &\quad \text{a \ttt{(\(\to\) Nat \(\mathcal{U}\))}} \\
    \ttt{base} &\quad \text{a \ttt{(mot zero)}} \\
    \ttt{step} &\quad \text{a \ttt{(\(\Pi\) ((k Nat)) (\(\to\) (mot k) (mot (add1 k))))}}
\end{align*}
the expression \(E \coloneqq \ttt{(ind-Nat target mot base step)}\) is a \ttt{(mot target)}.
\\ \\
Furthermore, \(E\) is the same \ttt{(mot zero)} as \ttt{base} when \ttt{target} is zero, and the same \ttt{(mot (add1 n))} as \[
    \ttt{(step n (ind-Nat n mot base step))}
\] when \ttt{target} is \ttt{(add1 n)}.
\\ \\
Note that \ttt{ind-Nat} and \ttt{rec-Nat} compute in exactly the same way.
\\ \\
Roughly, \ttt{ind-Nat} can be used to compute expressions of the form \[
    f_n(n, f_{n - 1}(n-1, f_{n-2}(n - 2, \dots f_0(0, s)\dots)))
\] where \(s \in X_0\) and \(f_k : \bN \times X_k \to X_{k + 1}\). This isn't exactly right, though, as the functions \(f_k\) aren't exactly independent, as the notation might imply, but rather they all have ``roughly the same behaviour". \textcolor{red}{How can we describe this restriction on the sequence of functions more precisely?}
\\ \\
So, we can think of applying \ttt{ind-Nat} to \(n\) as a sort of path through the first \(n + 1\) types in an infinite sequence of types. It seems that not all sequences of types are valid, though; for example, I don't think it is possible to use \ttt{ind-Nat} to traverse the sequence \ttt{Atom, Nat, Atom, Nat, \dots} \textcolor{red}{What kind of sequences are valid?}
\\ \\
In general, \textcolor{red}{what can \ttt{ind-Nat} do and not do? On one hand, it seems like the new \ttt{mot} argument makes \ttt{ind-Nat} significantly stronger, but it feels like there are some subtle restrictions that aren't obvious from the type}.
\subsection{Superfluence of \ttt{rec-Nat} with respect to \ttt{ind-Nat}}
\ttt{rec-Nat} is just \ttt{ind-Nat} with a constant function supplied for \ttt{mot}. This makes sense, as the \ttt{mot} parameter determines the type of \ttt{step} and the ``almost result" at each iteration, which in the case of \ttt{rec-Nat} never changes.
\subsection{Depending on \ttt{ind-Nat}}
\ttt{ind-Nat} is necessary when the type of our computation depends on the natural \ttt{target}; that is, \ttt{ind-Nat} is necessary when we are dealing with dependent types. So far, the only dependent type we have seen is the \ttt{Vec} type. By extension, any type which contains the \ttt{Vec} type, such as pairs of vectors, or functions on vectors, are also dependent.
\\ \\
In particular, \ttt{ind-Nat} is a natural choice when the value of a computation at the \((n + 1)\)'st stage can be expressed in terms of the value at the \(n\)'th stage \textit{and the types of these values are different}. This enables us to write algorithms like \hyperref[code:drop]{drop} (and hence \hyperref[code:nth-element]{nth-element}), \hyperref[code:range]{range}, \hyperref[code:compose]{compose}, \hyperref[code:args-to-vec]{args\(\to\)vec}, \hyperref[code:apply+]{apply+}, \hyperref[code:zip-with]{zip-with} and \hyperref[code:transpose]{transpose}.
\subsection{Equivalent formulations of induction}
The inductive principle embodied in \ttt{ind-Nat} is \textit{weak induction}, i.e the implication: for \(S\) a set, if
\begin{itemize}
    \item \(0 \in S\); and
    \item \(\forall n \in \bb N. (n \in S) \implies (n + 1 \in S)\)
\end{itemize}
then, \(\forall n \in \bb N. n \in S\).
\\ \\
There is also \textit{strong induction}, i.e the implication: for \(S\) a set, if \[
    \forall n \in \bb N. (\forall i \in \bb N. (i < n) \implies i \in S) \implies n \in S
\]
then \(\forall n \in \bb N. n \in S\).
\\ \\
It's well known that weak induction and strong induction are equivalent. \textcolor{red}{Does this equivalence carry over into Pie as well? Can we write a \ttt{strong-ind-Nat} eliminator, in which the result at stage \(n\) can depend on the result at any previous stage?} 
\section{Pick a Number, Any Number}
\subsection{Summary}
This chapter introduces the concept of reading types as statements. To this end, we see the \ttt{=} type constructor for the first time, which encodes the idea of equality into a type. We learn about using the \ttt{=} type, its sole constructor \ttt{same}, and one of its eliminators \ttt{cong}.
\subsection{The \ttt{=} type}
When \ttt{X} is a type and \ttt{from}, \ttt{to} are both \ttt{X}'s, the expression \ttt{(= X from to)} is a type.
\\ \\
The \ttt{=} type has one constructor: \ttt{same}. Given an \ttt{e} of type \ttt{X}, \ttt{(same e)} is a \ttt{(= X e e)}.
\\ \\
\ttt{cong} is an eliminator for \ttt{=}. Given a \ttt{target} of type \ttt{(= X from to)}, and a function \ttt{f} of type \ttt{(\(\to\) X Y)}, \ttt{(cong target f)} returns a value of type \ttt{(= Y (f from) (f to))}. \ttt{cong} represents the well-definedness of functions; that is, if \(x = y\) then \(f(x) = f(y)\). \ttt{(cong (same x) f)} computes as \ttt{(same (f x))}.
\subsection{Reading types as statements}
We may think of types as statements and values as proofs of those statements. Most types we have seen so far represent quite uninteresting statements, such as \ttt{Nat}, \ttt{Atom}, \ttt{List}, and \ttt{Vec}.
\\ \\
The \(\Pi\) type constructor is quite interesting, as it represents the universal quantifier. That is, a type like \[\ttt{(\(\Pi\) ((n Nat)) (= Nat (+ n n) (* 2 n)))}\] represents the statement ``for all \ttt{Nat} \(n\), \ttt{(+ n n)} equals \ttt{(* 2 n)}".
\\ \\
The \(\to\) type constructor represents the implication; i.e \ttt{(\(\to\) X Y)} represents the statement ``if \ttt{X}, then \ttt{Y}". The connection with \(\Pi\) here is interesting; we know that \(\to\) is syntax sugar for \(\Pi\). So, we might read a \(\Pi\) type as a statement of the form \(\forall x \in X. P(x)\). Then, in a \(\to\) type, \(P\) doesn't use \(x\), so the statement is really \(\forall x \in X. P\). 
\\ \\
False statements, then, are types for which no values exist. An example is the type \ttt{(= Nat 3 4)}.
\subsection{Some questions}
\subsubsection*{Existence statements}
\textcolor{red}{How do we represent more sophisticated existence statements?} In some sense, the basic data types like \ttt{Nat}, \ttt{Atom} etc. are existence statements, but they're too simple. \textcolor{red}{How do we something more sophisticated like \(\exists x. P(x)\)?} 
\subsubsection*{Negation}
\textcolor{red}{What does negation look like? How do we negate a type?}
\subsubsection*{Judgements vs Expressions}
The authors make a big deal of distinguishing between judgements which are ``attitudes one takes towards expressions" and what types like the \ttt{=} - which are themselves expressions - mean. I don't quite understand this distinction. \textcolor{red}{What exactly is the difference between what the \ttt{=} type says and a judgment of the form ``\(x\) is the same \(X\) as \(y\)?"} It seems like ``sameness" (in the judgement sense) is a strictly stronger condition than equality (in the \ttt{=} sense), so that two things that are the same are necessarily equal but not vice versa.
\subsubsection*{Using proofs}
How do we make use of a proof? For example, The \hyperref[code:concatvec]{concatvec} implementation does not work if we switch the arguments to \ttt{+} in the type; \textcolor{red}{how can we use a proof of the commutativity of \ttt{+} to make such an alternative implementation work?}
\newpage
\begin{appendix}
\section{Some Cool Pie Code} \label{pie-code-appendix}
\renewcommand{\contentsname}{\normalsize Contents}
\localtableofcontents
\newpage\noindent
��This code listing was generated automatically from \href{https://github.com/thechosenreader/The-Little-Typer-Notes}{\ttt{lib.pie} in this repository}.

\subsection{Types}



\subsubsection{NaryTo} \label{code:NaryTo}

\begin{lstlisting}

; creates the type of an n-ary function

; where the arguments are of type a

; and the result is of type b

(claim NaryTo (-> Nat U U U))

(define NaryTo

    (lambda (n a b)

        (iter-Nat n b (lambda (u) (-> a u)))))

\end{lstlisting}



\subsubsection{NaryOp} \label{code:NaryOp}

\begin{lstlisting}

; creates the type of an n-ary operation over some type a

(claim NaryOp (-> Nat U U))

(define NaryOp (lambda (n a) (NaryTo n a a)))

\end{lstlisting}



\subsubsection{UNat (unary operator on Nat)} \label{code:UNat}

\begin{lstlisting}

(claim UNat U)

(define UNat (NaryOp 1 Nat))

\end{lstlisting}



\subsubsection{Predicate} \label{code:Predicate}

\begin{lstlisting}

(claim Predicate (-> U U))

(define Predicate (lambda (a) (-> a Nat)))

\end{lstlisting}



\subsubsection{BinNat (binary operator on Nat)} \label{code:BinNat}

\begin{lstlisting}

(claim BinNat U)

(define BinNat (NaryOp 2 Nat))

\end{lstlisting}





\subsection{Combinators}



\subsubsection{S (substitution)} \label{code:S}

\begin{lstlisting}

(claim S

    (Pi ((a U) (b U) (c U)) (-> (-> a b c) (-> a b) a c)))

(define S

    (lambda (a b c) (lambda (x y z) (x z (y z)))))

\end{lstlisting}



\subsubsection{K (const)} \label{code:K}

\begin{lstlisting}

(claim const

    (Pi ((a U) (b U)) (-> a b a)))

(define const

    (lambda (a b) (lambda (x y) x)))

\end{lstlisting}



\subsubsection{I (identity)} \label{code:I}

\begin{lstlisting}

(claim id 

    (Pi ((a U)) (-> a a)))

(define id

    (lambda (a) (lambda (x) x)))

\end{lstlisting}



\subsubsection{C (flip)} \label{code:C}

\begin{lstlisting}

(claim flip

    (Pi ((a U) (b U) (c U)) (-> (-> a b c) (-> b a c))))

(define flip

    (lambda (a b c)

        (lambda (f)

            (lambda (x y) (f y x)))))

\end{lstlisting}



\subsubsection{B (compose)} \label{code:B}

\begin{lstlisting}

(claim B

    (Pi ((a U) (b U) (c U)) (-> (-> b c) (-> a b) (-> a c))))

(define B

    (lambda (a b c)

        (lambda (f g)

            (lambda (x) (f (g x))))))

\end{lstlisting}



\subsubsection{B1 (compose unary with binary)} \label{code:B1}

\begin{lstlisting}

(claim B1

    (Pi ((a U) (b U) (c U) (d U)) (-> (-> c d) (-> a b c) (-> a b d))))

;TODO: is it possible to use the B1 = BBB construction?

(define B1

    (lambda (a b c d)

        (lambda (f g)

            (lambda (x y) (f (g x y))))))

\end{lstlisting}





\subsection{Functional Tools}



\subsubsection{iterate-n} \label{code:iterate-n}

\begin{lstlisting}

(claim iterate-n

    (Pi ((a U)) (-> (-> a a) a Nat a)))

(define iterate-n

    (lambda (a)

        (lambda (f s n)

            (iter-Nat n s f))))

\end{lstlisting}



\subsubsection{projection} \label{code:projection}

\begin{lstlisting}

(claim __add-argument-end

    (Pi ((k Nat) (a U)) (-> (NaryOp k a) (NaryOp (add1 k) a))))

(define __add-argument-end

    (lambda (k a)

        (ind-Nat

            k

            (lambda (i) (-> (NaryOp i a) (NaryOp (add1 i) a)))

            (lambda (f x) f)

            (lambda (k-1 almost) 

                (lambda (f x) (almost (f x)))))))

; creates an (i + j + 1)-ary function on the type a 

;  which returns its i'th argument (counting from 0)

(claim projection

    (Pi ((i Nat)

         (j Nat)

         (a U))

         ;       this is just the '+' function

         ;       -----------------------------------

        (NaryOp (iterate-n Nat (lambda (n) (add1 n)) (add1 j) i) a)))

(define projection

    (lambda (i j a)

        (ind-Nat

            i

            ;                    this is just the '+' function

            ;                    -----------------------------------

            (lambda (k) (NaryOp (iterate-n Nat (lambda (n) (add1 n)) (add1 j) k) a))

            (ind-Nat

                j

                (lambda (k) (NaryOp (add1 k) a))

                (lambda (x) x)

                (lambda (k-1 almost) (__add-argument-end (add1 k-1) a almost)))

            (lambda (k-1 almost) (lambda (x) almost)))))

\end{lstlisting}



\subsubsection{compose} \label{code:compose}

\begin{lstlisting}

; composes a unary function from b to c with a k-ary function from a^k to b

(claim compose

    (Pi 

        ((k Nat)

         (a U)

         (b U)

         (c U))

        (-> (-> b c) (NaryTo k a b) (NaryTo k a c))))

(define compose

    (lambda (k a b c)

        (ind-Nat

            k

            (lambda (i) (-> (-> b c) (NaryTo i a b) (NaryTo i a c)))

            (lambda (f) f)

            (lambda (k-1 compose-k-1)

                (lambda (f g x) (compose-k-1 f (g x))))

            )))

\end{lstlisting}



\subsubsection{args-to-vec} \label{code:args-to-vec}

\begin{lstlisting}

; constructs function from a^{k+1} -> (Vec a (k + 1)) which simply collects 

; its arguments into a vector

(claim args->vec

    (Pi 

        ((a U) 

         (k Nat)) 

        (NaryTo k a (Vec a k))))

(define args->vec

    (lambda (a k)

        (ind-Nat

            k

            (lambda (i) (NaryTo i a (Vec a i)))

            vecnil

            (lambda (k-1 g)

                (lambda (x)

                    (compose

                        k-1

                        a (Vec a k-1) (Vec a (add1 k-1))

                        (lambda (v) (vec:: x v))

                        g

                        ))))))



; cleaner alias

(claim make-vec

    (Pi 

        ((a U) 

         (k Nat)) 

        (NaryTo k a (Vec a k))))

(define make-vec args->vec)

\end{lstlisting}



\subsubsection{args-to-lst} \label{code:args-to-lst}

\begin{lstlisting}

; constructs function from a^{k+1} -> (List a) which simply collects 

; its arguments into a list

(claim args->lst

  (Pi 

    ((a U) 

     (k Nat)) 

    (NaryTo k a (List a))))

(define args->lst

    (lambda (a k)

        (ind-Nat

            k

            (lambda (i) (NaryTo i a (List a)))

            nil

            (lambda (k-1 g)

                (lambda (x)

                    (compose

                        k-1

                        a (List a) (List a)

                        (lambda (v) (:: x v))

                        g))))))



; this doesn't work cause vec->lst is defined later

;  and i'm too lazy to move stuff around



; (define args->lst

;   (lambda (a k) 

;     (compose k 

;         a (Vec a (add1 k)) (List a) 

;         (vec->lst a (add1 k)) 

;         (args->vec a k))))



; cleaner alias

(claim make-lst

  (Pi 

    ((a U) 

     (k Nat)) 

    (NaryTo k a (List a))))

(define make-lst args->lst)

\end{lstlisting}



\subsubsection{apply+ (vec-to-args)} \label{code:apply+}

\begin{lstlisting}

; applies a function from a^{k + 1} -> b to a (Vec a (k + 1))

(claim apply+

    (Pi

        ((a U)

         (b U)

         (k Nat))

        (-> (NaryTo (add1 k) a b) (Vec a (add1 k)) b)))

(define apply+

    (lambda (a b k)

        (ind-Nat

            k

            (lambda (i) (-> (NaryTo (add1 i) a b) (Vec a (add1 i)) b))

            (lambda (g v) (g (head v)))

            (lambda (k-1 apply+k-1)

                (lambda (g v) (apply+k-1 (g (head v)) (tail v)))))))

\end{lstlisting}



\subsubsection{zip-with} \label{code:zip-with}

\begin{lstlisting}

(claim zip-with

  (Pi

    ((a U)

     (b U)

     (c U)

     (k Nat))

    (-> (-> a b c) (Vec a k) (Vec b k) (Vec c k))))

(define zip-with

  (lambda (a b c k)

    (ind-Nat

      k

      (lambda (i) (-> (-> a b c) (Vec a i) (Vec b i) (Vec c i)))

      (lambda (f v w) vecnil)

      (lambda (k-1 almost-zip-with)

        (lambda (f v w)

          (vec:: (f (head v) (head w)) (almost-zip-with f (tail v) (tail w))))))))

\end{lstlisting}



\subsubsection{zip (zip-with cons)} \label{code:zip}

\begin{lstlisting}

(claim zip

  (Pi

    ((a U)

     (b U)

     (k Nat))

    (-> (Vec a k) (Vec b k) (Vec (Pair a b) k))))

(define zip

  (lambda (a b k)

    (zip-with a b (Pair a b) k (lambda (x y) (cons x y)))))

\end{lstlisting}





\subsection{Utilities}



\subsubsection{successor} \label{code:successor}

\begin{lstlisting}

(claim succ UNat)

(define succ (lambda (n) (add1 n)))

\end{lstlisting}



\subsubsection{predecessor} \label{code:predecessor}

\begin{lstlisting}

(claim pred UNat)

(define pred (lambda (n) (which-Nat n zero (id Nat))))

\end{lstlisting}



\subsubsection{sgn} \label{code:sgn}

\begin{lstlisting}

; 1 if n > 0, 0 otherwise

(claim sgn UNat)

(define sgn (lambda (n) (which-Nat n zero ((const Nat Nat) 1))))

\end{lstlisting}



\subsubsection{cosgn} \label{code:cosgn}

\begin{lstlisting}

; 0 if n > 0, 1 otherwise

(claim cosgn UNat)

(define cosgn (lambda (n) (which-Nat n 1 ((const Nat Nat) 0))))

\end{lstlisting}



\subsubsection{mu (bounded minimization)} \label{code:mu}

\begin{lstlisting}

; the smallest natural z <= x satisfying p(z), or x + 1 if no such z exists

(claim mu (-> Nat (Predicate Nat) Nat))

(define mu

    (lambda (x p)

        ; start at 0 and increment iff p evaluates to false (i.e 0)

        (iter-Nat x 

            (cosgn (p 0)) ; // 1 if p(0) is false and 0 if p(0) is true

            (lambda (z) ((iterate-n Nat) succ z (cosgn (p z)))))))

\end{lstlisting}





\subsection{Arithmetic \& Comparisons}



\subsubsection{addition} \label{code:addition}

\begin{lstlisting}

(claim + BinNat)

(define + ((iterate-n Nat) succ))

\end{lstlisting}



\subsubsection{subtraction} \label{code:subtraction}

\begin{lstlisting}

(claim - BinNat)

(define - ((iterate-n Nat) pred))

\end{lstlisting}



\subsubsection{multiplication} \label{code:multiplication}

\begin{lstlisting}

(claim * BinNat)

(define * (lambda (n) ((iterate-n Nat) (+ n) 0)))

\end{lstlisting}



\subsubsection{exponentiation} \label{code:exponentiation}

\begin{lstlisting}

(claim ^ BinNat)

(define ^ (lambda (n m) ((iterate-n Nat) (* n) 1 m)))

\end{lstlisting}



\subsubsection{equality} \label{code:equality}

\begin{lstlisting}

; output 1 when n = m and 0 otherwise

(claim == BinNat)

(define ==

  (lambda (n m)

    ; (n - m) + (m - n) = 0 if n = m

    ; and > 0 otherwise, hence subtracting

    ; it from 1 returns 1 iff n = m

    (- 1 (+ (- n m) (- m n)))))

\end{lstlisting}



\subsubsection{gt} \label{code:gt}

\begin{lstlisting}

(claim > BinNat)

(define >

    (lambda (n m) (sgn (- n m))))

\end{lstlisting}



\subsubsection{lt} \label{code:lt}

\begin{lstlisting}

(claim < BinNat)

(define <

    (lambda (n m) (sgn (- m n))))

\end{lstlisting}



\subsubsection{gte} \label{code:gte}

\begin{lstlisting}

(claim >= BinNat)

(define >=

    (lambda (n m) (cosgn (< n m))))

\end{lstlisting}



\subsubsection{lte} \label{code:lte}

\begin{lstlisting}

(claim <= BinNat)

(define <=

    (lambda (n m) (cosgn (> n m))))

\end{lstlisting}



\subsubsection{floordiv} \label{code:floordiv}

\begin{lstlisting}

; floor(n/m) is the smallest number z such that (z + 1)m > n

(claim // BinNat)

(define //

    (lambda (n m)

        (mu n (lambda (z) (> (* (succ z) m) n)))))

\end{lstlisting}



\subsubsection{modulus} \label{code:modulus}

\begin{lstlisting}

(claim % BinNat)

(define %

    (lambda (n m) (- n (* m (// n m)))))

\end{lstlisting}



\subsubsection{divides?} \label{code:divides?}

\begin{lstlisting}

(claim divides? BinNat)

(define divides?

    (lambda (n m) (== 0 (% m n))))

\end{lstlisting}



\subsubsection{lo (get exponent of factor)} \label{code:lo}

\begin{lstlisting}

; lo is the b-adic valuation of n

; it is the smallest integer z such that b^(z + 1) does not divide n

(claim lo BinNat)

(define lo

    (lambda (b n) (mu n (lambda (z) (cosgn (divides? (^ b (succ z)) n))))))

\end{lstlisting}



\subsubsection{sum} \label{code:sum}

\begin{lstlisting}

; sum a function f from 0 to n

; i.e, f(0) + f(1) + ... + f(n)

(claim sum (-> UNat Nat Nat))

(define sum

    (lambda (f n)

        (rec-Nat n (f 0) (lambda (n-1 s) (+ s (f (succ n-1)))))))

\end{lstlisting}



\subsubsection{prod} \label{code:prod}

\begin{lstlisting}

; multiply a function f from 0 to n

; i.e, f(0) * f(1) * ... * f(n)

(claim prod (-> UNat Nat Nat))

(define prod

    (lambda (f n)

        (rec-Nat n (f 0) (lambda (n-1 s) (* s (f (succ n-1)))))))

\end{lstlisting}





\subsection{Logic}



\subsubsection{not} \label{code:not}

\begin{lstlisting}

(claim not UNat)

(define not cosgn)

\end{lstlisting}



\subsubsection{and} \label{code:and}

\begin{lstlisting}

(claim and BinNat)

(define and *)

\end{lstlisting}



\subsubsection{or} \label{code:or}

\begin{lstlisting}

(claim or BinNat)

(define or

    (lambda (p q) (sgn (+ p q))))

\end{lstlisting}



\subsubsection{xor} \label{code:xor}

\begin{lstlisting}

(claim xor BinNat)

(define xor

    (lambda (p q) (== 1 (+ p q))))

\end{lstlisting}





\subsection{Proofs (Arithmetic)}



\subsubsection{+associates (associativity of multiplication)} \label{code:+associates}

\begin{lstlisting}

; proof that (+ a (+ b c)) = (+ (+ a b) c) for all a, b, c in Nat

(claim +associates

  (Pi ((a Nat) (b Nat) (c Nat))

    (= Nat (+ a (+ b c)) (+ (+ a b) c))))

(define +associates

  (lambda (a b c)

    (ind-Nat

      c

      (lambda (i) (= Nat (+ a (+ b i)) (+ (+ a b) i)))

      (same (+ a b))

      (lambda (k-1 almost)

        (cong almost succ)))))

\end{lstlisting}



\subsubsection{+commutes (commutativity of addition)} \label{code:+commutes}

\begin{lstlisting}

; proof that (+ m n) = (+ n m) for all m, n in Nat

(claim +0-commutes

  (Pi ((m Nat)) (= Nat (+ m 0) (+ 0 m))))

(define +0-commutes

  (lambda (m)

    (ind-Nat

      m

      (lambda (i) (= Nat (+ i 0) (+ 0 i)))

      (same 0)

      (lambda (k-1 almost) (cong almost succ)))))



(claim +1-associates

  (Pi ((m Nat) (n Nat))

    (= Nat (+ (add1 m) n) (+ m (add1 n)))))

(define +1-associates

  (lambda (m n)

    (ind-Nat

      n

      (lambda (i) (= Nat (+ (add1 m) i) (+ m (add1 i))))

      (same (add1 m))

      (lambda (n-1 almost)

        (cong almost succ)))))



(claim +commutes

  (Pi ((m Nat) (n Nat)) (= Nat (+ m n) (+ n m))))

(define +commutes

  (lambda (m n)

    (ind-Nat

      n

      (lambda (i) (= Nat (+ m i) (+ i m)))

      (+0-commutes m)

      (lambda (k-1 almost)

        (trans

         (cong almost succ)

         (symm (+1-associates k-1 m)))))))

\end{lstlisting}



\subsubsection{*commutes (commutativity of multiplication)} \label{code:*commutes}

\begin{lstlisting}



; the algebraic identity (a + b) + (c + d) = (c + b) + (a + d)

(claim a+b_+_c+d=c+b_+_a+d

  (Pi ((a Nat) (b Nat) (c Nat) (d Nat))

    (= Nat (+ (+ a b) (+ c d)) (+ (+ c b) (+ a d)))))

(define a+b_+_c+d=c+b_+_a+d

  (lambda (a b c d)

    (trans

      (cong (+commutes a b) (the (-> Nat Nat ) (lambda (x) (+ x (+ c d)))))

      (trans

        (symm (+associates b a (+ c d)))

        (trans

          (cong (+commutes a (+ c d)) (+ b))

          (trans

            (cong (symm (+associates c d a)) (+ b))

            (trans

              (+associates b c (+ d a))

              (trans

                (cong (+commutes b c) (the (-> Nat Nat) (lambda (x) (+ x (+ d a)))))

                (cong (+commutes d a) (+ (+ c b)))))))))))



; the identity (m * (n + 1)) = m + (n * m)

(claim *distrib-over-1

  (Pi ((m Nat) (n Nat))

    (= Nat (* (+ n 1) m) (+ m (* n m)))))

(define *distrib-over-1

  (lambda (m n)

    (ind-Nat

      m

      (lambda (i) (= Nat (* (+ n 1) i) (+ i (* n i))))

      (same 0)

      (lambda (i n+1*i=i+n*i)

        (trans

          (cong n+1*i=i+n*i (+ (+ n 1)))

          (a+b_+_c+d=c+b_+_a+d n 1 i (* n i)))))))



; m * 0 = 0 * m for all m

(claim *0-commutes

  (Pi ((m Nat))

    (= Nat (* m 0) (* 0 m))))

(define *0-commutes

  (lambda (m)

    (ind-Nat

      m

      (lambda (i) (= Nat (* i 0) (* 0 i)))

      (same 0)

      (lambda (k-1 almost)

        (cong almost (+ 0))))))



; the identity (m * n) = (n * m)

(claim *commutes

  (Pi ((m Nat) (n Nat))

    (= Nat (* m n) (* n m))))

(define *commutes

  (lambda (m n)

    (ind-Nat

      n

      (lambda (i) (= Nat (* m i) (* i m)))

      (*0-commutes m)

      (lambda (i m*i=i*m)

        (trans

          (cong m*i=i*m (+ m))

          (symm (*distrib-over-1 m i)))))))

\end{lstlisting}



\subsubsection{*distributes-over+right (right distributivity of multiplication over addition \((a + b)c = ac + bc\))} \label{code:*distributes-over+right}

\begin{lstlisting}

(claim a+b_+_c+d=a+c_+_b+d

  (Pi ((a Nat) (b Nat) (c Nat) (d Nat))

    (= Nat (+ (+ a b) (+ c d)) (+ (+ a c) (+ b d)))))

(define a+b_+_c+d=a+c_+_b+d

  (lambda (a b c d)

    (trans

      (symm (+associates a b (+ c d)))

      (trans

        (cong (+associates b c d) (+ a))

        (trans

          (cong (+commutes b c) (the (-> Nat Nat) (lambda (x) (+ a (+ x d)))))

          (trans

            (cong (symm (+associates c b d)) (+ a))

            (+associates a c (+ b d))))))))



(claim *distributes-over+_right

  (Pi ((a Nat) (b Nat) (c Nat))

    (= Nat (* (+ a b) c) (+ (* a c) (* b c)))))

(define *distributes-over+_right

  (lambda (a b c)

    (ind-Nat

      c

      (lambda (i) (= Nat (* (+ a b) i) (+ (* a i) (* b i))))

      (same 0)

      (lambda (i a+b_*i=a*i+b*i)

        (trans

          (cong a+b_*i=a*i+b*i (+ (+ a b)))

          (a+b_+_c+d=a+c_+_b+d a b (* a i) (* b i)))))))

\end{lstlisting}



\subsubsection{*distributes-over+left (left distributivity of multiplication over addition \(a(b + c) = ab + ac\))} \label{code:*distributes-over+left}

\begin{lstlisting}

(claim *distributes-over+_left

  (Pi ((a Nat) (b Nat) (c Nat))

    (= Nat (* a (+ b c)) (+ (* a b) (* a c)))))

(define *distributes-over+_left

  (lambda (a b c)

    (trans

      (*commutes a (+ b c))

      (trans

        (*distributes-over+_right b c a)

        (trans

         (cong (*commutes b a) (the (-> Nat Nat) (lambda (x) (+ x (* c a)))))

         (cong (*commutes c a) (+ (* a b))))))))

\end{lstlisting}



\subsubsection{*associates (associativity of multiplication)} \label{code:*associates}

\begin{lstlisting}

(claim *associates

  (Pi ((a Nat) (b Nat) (c Nat))

    (= Nat (* a (* b c)) (* (* a b) c))))

(define *associates

  (lambda (a b c)

    (ind-Nat

      c

      (lambda (i) (= Nat (* a (* b i)) (* (* a b) i)))

      (same 0)

      (lambda (i a*_b*i=a*b_*i)

        (trans

          (*distributes-over+_left a b (* b i))

          (cong a*_b*i=a*b_*i (+ (* a b))))))))

\end{lstlisting}



\subsubsection{n+n=2n} \label{code:n+n=2n}

\begin{lstlisting}

(claim n+n=2n

  (Pi ((n Nat))

    (= Nat (+ n n) (* 2 n))))

(define n+n=2n

  (lambda (n)

    (ind-Nat

      n

      (lambda (i) (= Nat (+ i i) (* 2 i)))

      (same 0)

      (lambda (i i+i=2i)

        (trans

         (cong (+commutes i 1) (the (-> Nat Nat) (lambda (x) (+ x (+ i 1)))))

         (trans

          (cong (+commutes i 1) (+ (+ 1 i)))  ; (1 + i) + (i + 1) = (1 + i) + (1 + i)

          (trans

           (symm (+associates 1 i (+ 1 i)))   ; (1 + i) + (i + 1) = 1 + (i + (i + 1))

           (trans

            (cong (+commutes i (+ 1 i)) (+ 1))

            (trans

             (+associates 1 (+ 1 i) i)

             (trans

              (cong (+associates 1 1 i) (the (-> Nat Nat) (lambda (x) (+ x i))))

              (trans

               (symm (+associates (+ 1 1) i i))

               (cong i+i=2i (+ (+ 1 1))))))))))))))

\end{lstlisting}





\subsection{Lists}



\subsubsection{prepend (:: or cons)} \label{code:prepend}

\begin{lstlisting}

(claim prepend (Pi ((a U)) (-> a (List a) (List a))))

(define prepend (lambda (a x xs) (:: x xs)))

\end{lstlisting}



\subsubsection{snoc (flip append)} \label{code:snoc}

\begin{lstlisting}

(claim snoc (Pi ((a U)) (-> (List a) a (List a))))

(define snoc

    (lambda (a)

        (lambda (lst e)

            (rec-List

                lst

                (:: e nil)

                (lambda (x xs snoc-xs) (:: x snoc-xs))))))



(claim append (Pi ((a U)) (-> a (List a) (List a))))

(define append (lambda (a) (flip (List a) a (List a) (snoc a))))

\end{lstlisting}





\subsubsection{length} \label{code:length}

\begin{lstlisting}

(claim length (Pi ((a U)) (-> (List a) Nat)))

(define length

    (lambda (a)

        (lambda (lst)

            (rec-List

                lst

                0

                (lambda (e es les) (succ les))))))

\end{lstlisting}



\subsubsection{concat} \label{code:concat}

\begin{lstlisting}

(claim concat (Pi ((a U)) (NaryOp 2 (List a))))

(define concat

    (lambda (a)

        (lambda (l1 l2)

            (rec-List

                l1

                l2

                (lambda (x xs concat-xs) (:: x concat-xs))))))

\end{lstlisting}



\subsubsection{reverse} \label{code:reverse}

\begin{lstlisting}

(claim reverse (Pi ((a U)) (NaryOp 1 (List a))))

(define reverse

    (lambda (a lst)

        (rec-List lst

            (the (List a) nil)

            (lambda (x xs reverse-xs) (snoc a reverse-xs x)))))

\end{lstlisting}



\subsubsection{map} \label{code:map}

\begin{lstlisting}

(claim map (Pi ((a U) (b U)) (-> (-> a b) (List a) (List b))))

(define map

    (lambda (a b f lst)

        (rec-List

            lst

            (the (List b) nil)

            (lambda (x xs map-xs) (:: (f x) map-xs)))))

\end{lstlisting}



\subsubsection{filter} \label{code:filter}

\begin{lstlisting}

(claim filter (Pi ((a U)) (-> (Predicate a) (List a) (List a))))

(define filter

    (lambda (a p lst)

        (rec-List

            lst

            (the (List a) nil)

            (lambda (x xs filter-xs)

                (which-Nat (p x) filter-xs (lambda (n) (:: x filter-xs)))))))

\end{lstlisting}



\subsubsection{foldr} \label{code:foldr}

\begin{lstlisting}

(claim foldr (Pi ((a U) (b U)) (-> (-> a b b) b (List a) b)))

(define foldr

    (lambda (a b f s l)

        (rec-List 

            l

            s

            (lambda (x xs fold-xs) (f x fold-xs)))))

\end{lstlisting}



\subsubsection{foldl} \label{code:foldl}

\begin{lstlisting}

(claim foldl (Pi ((a U) (b U)) (-> (-> b a b) b (List a) b)))

(define foldl

    (lambda (a b f s l) (foldr a b (flip b a b f) s (reverse a l))))

\end{lstlisting}



\subsubsection{repeat} \label{code:repeat}

\begin{lstlisting}

(claim repeat (Pi ((a U)) (-> Nat a (List a))))

(define repeat

    (lambda (a k x) (iterate-n (List a) (prepend a x) (the (List a) nil) k)))

\end{lstlisting}





\subsubsection{lst-to-vec} \label{code:lst-to-vec}

\begin{lstlisting}

(claim lst->vec (Pi ((a U) (lst (List a))) (Vec a (length a lst))))

(define lst->vec

  (lambda (a lst)

    (ind-List

      lst

      (lambda (i) (Vec a (length a i)))

      vecnil

      (lambda (x xs vxs) (vec:: x vxs)))))

\end{lstlisting}



\subsection{Proofs (Lists)}

\subsubsection{map-commutes-with-compose} \label{code:map-commutes-with-compose}

\begin{lstlisting}

(claim map-commute-compose

  (Pi

    ((a U) (b U) (c U)

     (l (List a)) (f (-> a b)) (g (-> b c)))

    (= (List c) (map a c (lambda (x) (g (f x))) l) (map b c g (map a b f l)))))

(define map-commute-compose

  (lambda (a b c l f g)

    (ind-List

      l

      (lambda (i) (= (List c) (map a c (lambda (x) (g (f x))) i) (map b c g (map a b f i))))

      (same nil)

      (lambda (x xs almost)

        (cong almost (the (-> (List c) (List c)) (lambda (v) (:: (g (f x)) v))))))))

\end{lstlisting}



\subsection{Vectors}



\subsubsection{repeatvec} \label{code:repeatvec}

\begin{lstlisting}

(claim repeatvec

  (Pi ((a U) (k Nat)) (-> a (Vec a k))))

(define repeatvec

  (lambda (a k e)

    (ind-Nat

      k

      (lambda (i) (Vec a i))

      vecnil

      (lambda (k-1 almost) (vec:: e almost)))))

\end{lstlisting}



\subsubsection{vec-to-lst} \label{code:vec-to-lst}

\begin{lstlisting}

(claim vec->lst (Pi ((a U) (k Nat)) (-> (Vec a k) (List a))))

(define vec->lst

  (lambda (a k v)

    (ind-Vec k v

      (lambda (j w) (List a))

      nil

      (lambda (k-1 e es les) (:: e les)))))

\end{lstlisting}



\subsubsection{drop} \label{code:drop}

\begin{lstlisting}

; remove first n elements from a vector of length n + k

(claim drop (Pi ((a U) (n Nat) (k Nat)) (-> (Vec a (+ k n)) (Vec a k))))

(define drop

  (lambda (a n k)

    (ind-Nat

      n

      (lambda (i) (-> (Vec a (+ k i)) (Vec a k)))

      (lambda (x) x)

      (lambda (n-1 f) (lambda (w) (f (tail w)))))))

\end{lstlisting}



\subsubsection{nth-element} \label{code:nth-element}

\begin{lstlisting}

(claim nth-element (Pi ((a U) (i Nat) (k Nat)) (-> (Vec a (+ (add1 k) i)) a)))

(define nth-element

  (lambda (a i k v) (head (drop a i (add1 k) v))))

\end{lstlisting}



\subsubsection{range} \label{code:range}

\begin{lstlisting}

; the sequence (0, ..., k-1)

(claim range (Pi ((k Nat)) (Vec Nat k)))

(define range

  (lambda (k)

    (ind-Nat

      k

      (lambda (n) (Vec Nat n))

      vecnil

      (lambda (k-1 rk-1) (vec:: (- k (succ k-1)) rk-1)))))



; the sequence (s, ..., e-1)

(claim range2 (Pi ((s Nat) (e Nat)) (Vec Nat (- e s))))

; (- m n) == 0 for n >= m

(define range2

  (lambda (s e)

    (ind-Nat

      (- e s)

      (lambda (n) (Vec Nat n))

      vecnil

      (lambda (k-1 rk-1) (vec:: (- e (succ k-1)) rk-1)))))



; the sequence {s + kst : k \in N, s + kst < e} (ordered by k)

(claim range3 (Pi ((s Nat) (e Nat) (st Nat)) (Vec Nat (succ (// (- e (succ s)) st)))))

(define range3

  (lambda (s e st)

    (ind-Nat

      (succ (// (- e (succ s)) st))

      (lambda (n) (Vec Nat n))

      vecnil

      (lambda (k-1 rk-1) (vec:: (+ s (* st (- (succ (// (- e (succ s)) st)) (succ k-1)))) rk-1)))))

\end{lstlisting}



\subsubsection{mapvec} \label{code:mapvec}

\begin{lstlisting}

(claim mapvec (Pi ((a U) (b U) (k Nat)) (-> (-> a b) (Vec a k) (Vec b k))))

(define mapvec

  (lambda (a b k f v)

    (ind-Vec k v

      (lambda (j v) (Vec b j))

      vecnil

      (lambda (k-1 e es mes) (vec:: (f e) mes)))))

\end{lstlisting}



\subsubsection{concatvec} \label{code:concatvec}

\begin{lstlisting}

(claim concatvec (Pi ((a U) (k Nat) (j Nat)) (-> (Vec a k) (Vec a j) (Vec a (+ j k)))))

(define concatvec

  (lambda (a k j v w)

    (ind-Vec k v

      (lambda (i x) (Vec a (+ j i)))

      w

      (lambda (k-1 e es ces) (vec:: e ces)))))

\end{lstlisting}



\subsubsection{appendvec} \label{code:appendvec}

\begin{lstlisting}

(claim appendvec (Pi ((a U) (k Nat)) (-> a (Vec a k) (Vec a (add1 k)))))

(define appendvec

  (lambda (a k e v)

    (ind-Vec k v

      (lambda (i x) (Vec a (add1 i)))

      (vec:: e vecnil)

      (lambda (k-1 e es ces) (vec:: e ces)))))

\end{lstlisting}



\subsubsection{reversevec} \label{code:reversevec}

\begin{lstlisting}

(claim reversevec (Pi ((a U) (k Nat)) (-> (Vec a k) (Vec a k))))

(define reversevec

  (lambda (a k v)

    (ind-Vec k v

      (lambda (i x) (Vec a i))

      vecnil

      (lambda (k-1 e es ces) (appendvec a k-1 e ces)))))

\end{lstlisting}



\subsubsection{transpose} \label{code:transpose}

\begin{lstlisting}

(claim transpose

  (Pi ((m Nat)

       (n Nat)

       (a U))

    (-> (Vec (Vec a n) m) (Vec (Vec a m) n))))

(define transpose

  (lambda (m n a)

    (ind-Nat

      m

      (lambda (i) (-> (Vec (Vec a n) i) (Vec (Vec a i) n)))

      (lambda (M) (repeatvec (Vec a 0) n vecnil))

      (lambda (m-1 almost-transpose)

        (lambda (M)

          (zip-with

            a (Vec a m-1) (Vec a (add1 m-1))

            n

            (lambda (x row) (vec:: x row))

            (head M)

            (almost-transpose (tail M))))))))

; alternate approach inspired by the python code

;  transpose = lambda mat: tuple(zip(*mat))

(claim transpose2

  (Pi ((m Nat)

       (n Nat)

       (a U))

    (-> (Vec (Vec a n) m) (Vec (Vec a m) n))))

(define transpose2

  (lambda (m n a v)

    (ind-Vec

      m

      v

      (lambda (i v) (Vec (Vec a i) n))

      (repeatvec (Vec a 0) n vecnil)

      (lambda (k-1 r rs almost)

        (zip-with a (Vec a k-1) (Vec a (add1 k-1)) n 

            (lambda (e x) (vec:: e x)) 

            r almost)))))

\end{lstlisting}



\subsubsection{subsets} \label{code:subsets}

\begin{lstlisting}

(claim double-gets-next-power2

  (Pi ((n Nat)) (= Nat (^ 2 (add1 n)) (+ (^ 2 n) (^ 2 n)))))

(define double-gets-next-power2

  (lambda (n)

    (symm (n+n=2n (^ 2 n)))))



(claim subsets

  (Pi ((a U) (k Nat))

    (-> (Vec a k) (Vec (List a) (^ 2 k)))))

(define subsets

  (lambda (a k v)

    (ind-Vec k v

      (lambda (i w) (Vec (List a) (^ 2 i)))

      (vec:: nil vecnil)

      (lambda (i x xs subsets-of-xs)

        (replace

          (symm (double-gets-next-power2 i))

          (lambda (j) (Vec (List a) j))

          (concatvec (List a) (^ 2 i) (^ 2 i)

            (mapvec (List a) (List a) (^ 2 i) (lambda (l) (:: x l)) subsets-of-xs)

            subsets-of-xs))))))

\end{lstlisting}







\subsection{Control Flow}



\subsubsection{if} \label{code:if}

\begin{lstlisting}

; this isn't really a true if as it will always evaluate both branches

; (unless Pie is lazily evaluated) but it doesn't matter because

; Pie doesn't have any side effects 

(claim if (Pi ((a U)) (-> Nat a a a)))

(define if (lambda (a c true false) (which-Nat c false (const a Nat true))))

\end{lstlisting}



\subsubsection{cond} \label{code:cond}

\begin{lstlisting}

; the same caveat about evaluation as in "if" applies here

(claim cond (Pi ((a U)) (-> (List (Pair Nat a)) a a)))

(define cond

  (lambda (a cs else)

    (rec-List

      cs

      else

      (lambda (x xs cond-xs) (if a (car x) (cdr x) cond-xs)))))

\end{lstlisting}





\subsection{Miscellaneous programs}



\subsubsection{Fibonacci} \label{code:Fibonacci}

\begin{lstlisting}

; we want this function to satisfy

; g(n) = 2^{fib(n)}3^{fib(n + 1)}

; by using lo, we can solve for g(n + 1) in terms of g(n)

; then, fib(n) = lo(2, g(n))

(claim __fib-helper-g UNat)

(define __fib-helper-g

    (lambda (n+1)

        (rec-Nat n+1

            6

            (lambda (n g-n) 

                (*

                    (^ 2 (lo 3 g-n))

                    (^ 3

                       (+ (lo 3 g-n) (lo 2 g-n))))))))



(claim fib UNat)

(define fib (lambda (n) (lo 2 (__fib-helper-g n))))

\end{lstlisting}



\subsubsection{factorial} \label{code:factorial}

\begin{lstlisting}

(claim ! UNat)

(define ! (lambda (n) (prod succ (pred n))))

\end{lstlisting}



\subsubsection{prime?} \label{code:prime?}

\begin{lstlisting}

(claim prime? UNat)

; p is prime iff p has only 2 divisors, 1 and itself

(define prime? (lambda (p) (== 2 (sum (flip Nat Nat Nat divides? p) p))))

\end{lstlisting}



\subsubsection{nth-prime} \label{code:nth-prime}

\begin{lstlisting}

(claim nth-prime UNat)

; given the nth prime, we know the (n + 1)th prime

; is less than or equal to the factorial of the nth prime + 1

; so we can use mu

(define nth-prime

    (lambda (n+1)

        (rec-Nat

            n+1

            2

            (lambda (n pn) 

                (mu (succ (! pn)) 

                    (lambda (z) 

                        (and (prime? z) (> z pn))))))))

\end{lstlisting}



\subsubsection{rec-Nat2 (rec-Nat in terms of iter-Nat)} \label{code:rec-Nat2}

\begin{lstlisting}

(claim rec-Nat2 (Pi ((X U)) (-> Nat X (-> Nat X X) X)))

(define rec-Nat2

  (lambda (X target base step)

    (cdr (iter-Nat

           target

           (the (Pair Nat X) (cons 0 base))

           (lambda (p) (cons (succ (car p)) (step (car p) (cdr p))))))))

\end{lstlisting}



\subsubsection{ack (Ackermann function)} \label{code:ack}

\begin{lstlisting}

(claim ack (-> Nat Nat Nat))

(define ack

  (lambda (m)

    (iter-Nat

       m

       succ

       (lambda (almost-ack)

         (lambda (n) (iter-Nat (succ n) 1 almost-ack))))))

\end{lstlisting}



\subsubsection{replace-can-symm (symm in terms of replace)} \label{code:replace-can-symm}

\begin{lstlisting}

(claim replace-can-symm

  (Pi ((X U) (a X) (b X))

    (-> (= X a b) (= X b a))))

(define replace-can-symm

  (lambda (X a b a=b)

    (replace

      a=b

      (lambda (x) (= X x a))

      (same a))))

\end{lstlisting}



\subsubsection{replace-can-cong (cong in terms of replace)} \label{code:replace-can-cong}

\begin{lstlisting}

(claim replace-can-cong

  (Pi ((X U) (Y U) (from X) (to X) (f (-> X Y)))

    (-> (= X from to) (= Y (f from) (f to)))))

(define replace-can-cong

  (lambda (X Y from to f from=to)

    (replace

      from=to

      (lambda (x) (= Y (f from) (f x)))

      (same (f from)))))

\end{lstlisting}



\subsubsection{replace-can-trans (trans in terms of replace)} \label{code:replace-can-trans}

\begin{lstlisting}

(claim replace-can-trans

  (Pi ((X U) (from X) (middle X) (to X))

    (-> (= X from middle) (= X middle to) (= X from to))))

(define replace-can-trans

  (lambda (X from middle to from=middle middle=to)

    (replace

      middle=to

      (lambda (x) (= X from x))

      from=middle)))

\end{lstlisting}





\subsubsection{PairVec (the Vec type implemented with Pairs)} \label{code:PairVec}

\begin{lstlisting}

; PairVec ~ Vec

; (PairVec a k) is a list of k a's

; it gets expanded as (Pair a (Pair a (Pair a ... (k times) Trivial)))

(claim PairVec

  (Pi ((a U) (k Nat)) U))

(define PairVec

  (lambda (a k)

    (iter-Nat k

      Trivial ; (PairVec a 0) is Trivial

      (lambda (u) (Pair a u)))))



; pvnil is the constructor for (PairVec a 0)

(claim pv.nil Trivial)

(define pv.nil sole)



; pv:: is identical to the builtin cons

(claim pv:: (Pi ((a U) (k Nat)) (-> a (PairVec a k) (PairVec a (add1 k)))))

(define pv::

  (lambda (a k e p) (cons e p)))



; pv.head is identical to the builtin car

(claim pv.head

  (Pi ((a U) (k Nat)) (-> (PairVec a (add1 k)) a)))

(define pv.head

  (lambda (a k p) (car p)))



; pv.tail is identical to the builtin cdr

(claim pv.tail

  (Pi ((a U) (k Nat)) (-> (PairVec a (add1 k)) (PairVec a k))))

(define pv.tail

  (lambda (a k p) (cdr p)))



; a copy of the type of ind-Vec from the Pie reference

(claim ind-PairVec

  (Pi ((E U)

       (target-1 Nat)

       (target-2 (PairVec E target-1))

       (motive (Pi ((k Nat)) (-> (PairVec E k) U)))

       (base (motive zero pv.nil))

       (step (Pi ((k Nat)

                  (e E)

                  (es (PairVec E k)))

                  (-> (motive k es) (motive (add1 k) (cons e es))))))

    (motive target-1 target-2)))

; we use induction to create a function that can inductively eliminate

; a k+1-length PairVec from a function that can inductively eliminate

; a k-length Pairvec

(define ind-PairVec

    (lambda (E target-1 target-2 motive base step)

        ((ind-Nat

            target-1

            (lambda (i) (Pi ((w (PairVec E i))) (motive i w)))

            (lambda (w) base)

            (lambda (k-1 almost-ind)

                (lambda (w)

                    (step k-1 (car w) (cdr w) (almost-ind (cdr w))))))

        target-2)))

\end{lstlisting}






\end{appendix}
\end{document}
